% Question 2

% This is how to specify how many marks each part/question is worth 

\begin{question}
    Prove by induction, that for all positive integer \(n \), 

    \begin{questionparts}
        \item \(1^2 + 3^2 + \cdots + (2n-1)^{2} = \frac{1}{3}(4n^3 - n)\)
        \item Hence find, as a single natural logarithm, the value of \[ \Int{f(x)}{x, 0, \frac{1}{2}} \] \pts{3}
    \end{questionparts}
    
    \pts{Total 5 marks}
\end{question}

\begin{solution}
    % Use \textbf to refer to part numbers 
    Part \textbf{(i)} will be useful for part \textbf{(ii)}. 

    \begin{solutionparts}
        \item Suppose \(f(x) \) can be written as partial fractions i.e. 
        \[ \frac{4x}{1-x^4} = \frac{Ax }{1 -x^2} + \frac{Bx }{1 + x^2},  \] where \(A, B \in \mathbb{R }\). 
        \begin{align*}
            \implies 4x &= Ax(1+x^2) + Bx(1-x^2) \\
            &= (A-B)x^3 + (A+B)x \\
            \implies A - B = 0 ~~&\text{and}~~ A+B = 4 \\
            \implies A = ~&B = 2 \\
            \therefore f(x) &= \frac{2x}{1 - x^2} + \frac{2x}{1 + x^2}
        \end{align*}

        \item We can write \(f(x) \) using the partial fractions we found.
        
        \begin{align*}
            \Int{f(x)}{x, 0, \frac{1}{2}}  &= \Int{\frac{2x}{1 - x^2} + \frac{2x}{1 + x^2}}{x, 0, \frac{1}{2}}  \\
            &= \left[ - \ln \left| 1 - x^2 \right| + \ln \left|1 + x^2 \right|\right]_0^\frac{1}{2} \\
            &= \ln \left| \frac{1+ \left(\frac{1}{2}\right)^2}{1 - \left(\frac{1}{2}\right)^2}\right| \\
            &= \ln \frac{5}{3}.
        \end{align*}
        \newpage % Use \newpages to format longer solutions that span pages
        Which is the final answer. \kant[1]
    \end{solutionparts}
\end{solution}