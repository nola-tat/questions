\documentclass[a4paper, 11pt]{article}

\title{Title}

% LTeX: enabled=false

% LTeX: enabled=false

\usepackage{cool}

\usepackage[utf8]{inputenc}
\usepackage{latexsym,amsfonts,amssymb,amsthm,amsmath}

% Allows aligns to span pages
\allowdisplaybreaks


% Maths operators and definitions

\def\d{\mathrm{d}}

\DeclareMathOperator{\cosec}{cosec}
\DeclareMathOperator{\Var}{Var}

% LTeX: enabled=false
% cSpell:disable

\usepackage{kantlipsum}

% Coloured boxes for questions

\usepackage[most]{tcolorbox}


% Page size

\usepackage{geometry}
\geometry{a4paper, lmargin=2cm, rmargin=2.5cm, tmargin=3.5cm, bmargin=2.5cm}

% Paragraph spacing

% \usepackage{parskip}
% \setlength{\parindent}{3cm}


% Headers

\usepackage{titlesec}
\usepackage{titling}
\usepackage{titletoc}

\usepackage[titles]{tocloft}

% \setlength{\cftsecindent}{0pt}
\setlength{\cftsubsecindent}{0pt}
% \renewcommand\cftsecafterpnum{\vskip30pt}
\renewcommand\numberline[1]{}
% Changes style of section title

% Hides the section number 
\titleformat{\section}{}{}{0em}{}
% \titleformat{\subsection}{\normalfont\Large\bfseries}{Question \thesubsection}{0em}{}


\setlength{\parskip}{10pt}
\setlength{\parindent}{0pt}


% Changing contents page 

\renewcommand*{\contentsname}{}
\dottedcontents{section}[0em]{\normalfont\Large}{50em}{1pc}
% \setlength{\cftbeforesecskip}{5em}


% To do with question environment

\usepackage{enumerate}

\newlength{\qspace}
\setlength{\qspace}{15pt}

% Question environment
% Would like to try to get this to start a subsection called `Question qnumber'

\newcounter{qnumber}
\setcounter{qnumber}{1}


% Command to create a section with the qnumber in the title (unnumbered sections)
\newcommand{\newq}{\section[Problem \theqnumber]{Test}}


% Mind the awkward way that the spacing in section titles is accounted for
% Question environment
\newenvironment{question}%
{\newq\vspace{-1.25cm}\begin{tcolorbox}[title=\textbf{Problem \theqnumber}] % Change the formatting of tcolorbox title here
}
{
  \end{tcolorbox}\vspace{1em}
  \stepcounter{qnumber}
}

% Question Parts environment
\newenvironment{questionparts}[1][1]%
{
  \begin{enumerate}[\bfseries (i)]%
    \setcounter{enumii}{#1}
    \addtocounter{enumii}{-1}
    \setlength{\itemsep}{5mm}
    \setlength{\parskip}{8pt}
}
{
  \end{enumerate}
}

\newenvironment{questionsubparts}[1][1]%
{
  \begin{enumerate}[\bfseries (a)]%
    \setcounter{enumiii}{#1}
    \addtocounter{enumiii}{-1}
    \setlength{\itemsep}{5mm}
    \setlength{\parskip}{8pt}
}
{
  \end{enumerate}
}

%  Solution and solutionparts environments
\newenvironment{solution}{\textbf{Solution}\vspace{1em}\\}{}

\newenvironment{solutionparts}[1][1]%
{
  \begin{enumerate}[\bfseries (i)]%
    \setcounter{enumii}{2}
    \addtocounter{enumii}{-1}
    \setlength{\itemsep}{5mm}
    \setlength{\parskip}{8pt}
}
{
  \end{enumerate}
}

\newenvironment{solutionsubparts}[1][1]%
{
  \begin{enumerate}[\bfseries (a)]%
    \setcounter{enumii}{2}
    \addtocounter{enumii}{-1}
    \setlength{\itemsep}{5mm}
    \setlength{\parskip}{8pt}
}
{
  \end{enumerate}
}

% Command used for displaying how many points a question/questionpart is worth
\newcommand{\pts}[1]{\hfill \textbf{(#1)}}
% LTeX: enabled=false

\renewcommand{\comment}[1]{{\bf Comment} {\it #1}}
%\renewcommand{\comment}[1]{}

\newcommand{\bluecomment}[1]{{\color{blue}#1}}
%\renewcommand{\comment}[1]{}
\newcommand{\redcomment}[1]{{\color{red}#1}}
% LTeX: enabled=false

% Move title up
\setlength{\droptitle}{-5em}


% Formatting title/author/date
\pretitle{\begin{center}\huge}
\posttitle{\par\end{center}}

% \preauthor{\begin{center}}
% \postauthor{\end{center}}

\predate{\centering}
\postdate{\par}

\author{}
\date{\today}
\input{preamble/custom.tex}

% Loaded last
\usepackage[hidelinks]{hyperref}

% \renewcommand{\thesubsection}{}

\usepackage{tocloft}

% \setlength{\cftsecindent}{0pt}
\setlength{\cftsubsecindent}{0pt}

\renewcommand\numberline[1]{}


% Would like the question environment to start a section 
% Need to also fix the issue with solution parts


\begin{document}
    \pagenumbering{roman}
    \maketitle
    
    \tableofcontents

    \newpage
    \pagenumbering{arabic}
    \section{Question 1}

        \begin{question}
            Put the first question here.
            Use display maths to show large equations \[ f(x) = \E^{x} \]
        \end{question}

        \begin{solution}
            This is the solution. Hello world.

        \end{solution}


        
    \section{Question 2}

        \begin{question}
            The function \(f(x) \) is defined as \[ f(x) = \frac{4x}{1 - x^4}. \]

            \begin{questionparts}
                \item Express \(f(x) \) into partial fractions. 
                \item Hence find, as a single natural logarithm, the value of \[ \int_{0}^{\frac{1}{2}} f(x)~ \mathrm{d } x. \]
                \item As you can see this question has different parts
                \item \begin{questionparts}
                    \item test
                \end{questionparts}
            \end{questionparts}
            
        \end{question}

        \begin{solution}
            Type words possibly explaining initial thoughts about the question here.

            \begin{solutionparts}
                \item Suppose \(f(x) \) can be written as partial fractions i.e. 
                \[ \frac{4x}{1-x^4} = \frac{Ax }{1 -x^2} + \frac{Bx }{1 + x^2},  \] where \(A, B \in \mathbb{R }\). 
                \begin{align*}
                    \implies 4x &= Ax(1+x^2) + Bx(1-x^2) \\
                    &= (A-B)x^3 + (A+B)x \\
                    \implies A - B = 0 ~~&\text{and}~~ A+B = 4 \\
                    \implies A = ~&B = 2 \\
                    \therefore f(x) &= \frac{2x}{1 - x^2} + \frac{2x}{1 + x^2}
                \end{align*}

                \item We can write \(f(x) \) using the partial fractions we found.
                
                \begin{align*}
                    \int_{0}^{\frac{1}{2}} f(x) ~ \mathrm{d} x  &= \int_{0}^{\frac{1}{2}} \frac{2x}{1 - x^2} + \frac{2x}{1 + x^2} ~ \mathrm{d} x \\
                    &= \left[ - \ln \left| 1 - x^2 \right| + \ln \left|1 + x^2 \right|\right]_0^\frac{1}{2} \\
                    &= \ln \left| \frac{1+ \left(\frac{1}{2}\right)^2}{1 - \left(\frac{1}{2}\right)^2}\right| \\
                    &= \ln \frac{5}{3}.
                \end{align*}
                Which is the final answer. 
            \end{solutionparts}
        \end{solution}

\end{document}